\documentclass[times,specification,annotation,languages={russian,english}]{itmo-student-thesis}
\linespread{1.5}
%% Опции пакета:
%% - specification - если есть, генерируется задание, иначе не генерируется
%% - annotation - если есть, генерируется аннотация, иначе не генерируется
%% - times - делает все шрифтом Times New Roman, собирается с помощью xelatex
%% - languages={...} - устанавливает перечень используемых языков. По умолчанию это {english,russian}.
%%                     Последний из языков определяет текст основного документа.

%% Делает запятую в формулах более интеллектуальной, например:
%% $1,5x$ будет читаться как полтора икса, а не один запятая пять иксов.
%% Однако если написать $1, 5x$, то все будет как прежде.
\usepackage{icomma}

%% Один из пакетов, позволяющий делать таблицы на всю ширину текста.
\usepackage{tabularx}

%% Данные пакеты необязательны к использованию в бакалаврских/магистерских
%% Они нужны для иллюстративных целей
%% Начало
\usepackage{tikz}
\usetikzlibrary{arrows}
\usepackage{filecontents}
\begin{filecontents}{bachelor-thesis.bib}

\end{filecontents}
%% Конец

%% Указываем файл с библиографией.
\addbibresource{bachelor-thesis.bib}

\begin{document}

%% Оглавление
\tableofcontents

%% Макрос для введения. Совместим со старым стилевиком.
\startprefacepage
Graphs, native vs DSL queries, graph compression for big graphs, no current solutinos, SWH use case.

%% Начало содержательной части.
\chapter{Review}
\begin{enumerate}
    \item Domain intro, terminology
    \item What needs to be done, whats already done, why its not enough
    \item Task
    \item Conclusion
\end{enumerate}
Should contain the majority of citations and links. Further chapter should contain personal achievements.

\section{Querying graph data}
Graphs are useful, but native query implementations are cumbersome

\section{Gremlin graph traversal language}
Gremlin provides useful graph traversing abstractions

\section{Graph database limitations}
Gremlin is supported by graph DBs, but the are costly to scale

\section{Graph compression}
Instead of DBs one can use graph compression

\section{WebGraph limitations}
WebGraph has no Gremlin support, and limited property support. SWH use case.

\section{Task}\label{sec:task}
Implement TinkerPop, add easy property handling, analyse the performance

\chapter{Proposed solution}
\begin{enumerate}
    \item Subtasks
    \item Possible solutions
    \item Chosen solutions justification
\end{enumerate}

\section{Implementing TinkerPop for WebGraph}
Implement structural interfaces on top of WebGraph API.

\section{Universal property handling}
Tie WebGraph arc labels, and standard vertex props approaches to TinkerPop props.

\section{Performance analysis}
The bridge could introduce overhead. Choose a domain and a dataset, formulate queries, and profile them.

\chapter{Implementation}
\begin{enumerate}
    \item Implementation - the chosen architecture, encountered problems
    \item Comparison with existing solutions
    \item Practical usage - adoption act, how the solution was used in the company
\end{enumerate}
The chapter may be split, e.g. the comparsion with exisitng solutions is often separated.

\section{Implementing TinkerPop}
Specific interfaces, wrapper objects, QueryExecutor.

\section{Property handling}
Provider interface, default implementations with value providers.
Implemented standard approaches (vertex primitives, strings). Arc label abstraction, decomposition.

\chapter{Solution analysis}

\section{SWH}
Domain, dataset.

\section{Queries}
Formulated queries, gremlin representation.

\section{Benchmarker}
Profiles queries on a random sample. Hot/cold timing, native comparison, CSV aggregation.

\section{Test results}
Graphs, what they show.

\startconclusionpage
Short summary, which tasks were completed, and which goals reached. Future development if needed.

Library offering Gremlin support for any graph compressed WebGraph. Easy property handling, with standard approaches requiring minimal boilerplate.

Tests showed this and that.

\end{document}
